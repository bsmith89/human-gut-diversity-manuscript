\documentclass[12pt]{article}
\usepackage{manuscriptstyle}

\begin{document}
\title{\huge Why does the gut have so many species? \\
       {\Large The Paradox of the Plankton in human microbial ecology}}
\author{Byron J. Smith}
\date{November 2014}
\maketitle

\section{Introduction}
The average, healthy human gut is home to more than one thousand species of
bacteria and archaea~\citep{Claesson2009,Brestoff2013}.
This mega-diverse microbial assemblage has been shown to contribute to human
health in numerous ways:
fermentation in the large intestine produces a suite of short-chain fatty acids
which contribute to energy acquisition and regulation of the gut epithelium~%
\citep{Macfarlane2012};
both the adaptive and innate immune systems may be initially trained,
and subsequently calibrated,
by interactions with commensal bacteria in the gut~\citep{Brestoff2013};
and a healthy microbiome also appears to protect the host from certain
infections, notably \cdiff~\citep{Britton2012}.
A number of studies have correlated the level of species diversity in the gut
with disease states,
including obesity~\citep{Turnbaugh2008,LeChatelier2013},
% See HMP 2012 for this and...
inflammatory bowel diseases~\citep{Dicksved2008,Manichanh2006},
and recurrent \cdiff\ infections~\citep{Chang2008}.

Based on intuition, theoretical treatment%
~\citep{Lotka1925,Volterra1928,MacArthur1967a},
% TODO: Is Levin1970 a correct reference?
and empirical observation~\citep{Gause1932,Gause1936},
% TODO: Other empirical observations of Gause's Law?
the coexistence of two species which occupy
exactly the same niche should be untenable in the long term.
This is perhaps the only rule in ecology to be elevated to the
status of a natural law.
Gause's Law, named for the Russian biologist who contributed
heavily to its experimental description, is also called
the principle of competitive exclusion, since it is based on
the reciprocally negative interactions between two species which
depend on the same limiting resource.
It has been further refined to the statement: \(n\) species cannot coexist on
fewer than \(n\) resources, niches, or limiting factors%
~\citep[various authors, for a review see][]{Armstrong1980}.
% TODO: Maybe Armstrong isn't the best citations...he's kinda dismissive of
% competitive exclusion.
The growth of one population necessarily leads to the decline of
the other, ultimately leading to its extinction,
and at equilibrium, only the better competitor is able to persist.
Examples of strong, negative correlations,
or even co-exclusionary relationships,
between some species~\citep{Faust2012a} in the human microbiome
suggest that competitive exclusion is a relevant process.

While exceptions to this principle are frequently suggested%
~\citep{Koch1974a,Koch1974b},
% TODO: Is this reference correct?
and many cases of stable coexistence have been robustly described,
none have served to overthrow Gause's Law~\citep{Hutchinson1957}.
Instead, discovery of general mechanisms of coexistence, the
most important of which are described below, have served
to refine the definition of `niche', and improved the theoretical
understanding of competition.
Thanks to this policy of modifying definitions and theory
rather than lowering its status, Gause's Law has been widely described,
as a tautology~\citep{Hutchinson1961,Slobodkin1961}.
% TODO: Is Slobodkin1961 a good reference here?
This may be the case, although it has nonetheless served as a
rallying point for ecological discovery~\citep{Armstrong1980}.
How, given a finite number of limiting resources and an apparently
homogeneous environment, can more than 1000 species of bacteria
co-occur in the human gut?

% \paragraph{`The Paradox of the Plankton'}
The perceived contradiction between this central tenant
of community ecology
and the observation of high diversity in nature led
G. Evelyn Hutchinson to write
\article{The Paradox of the Plankton}~\citeyearpar{Hutchinson1961}.
In that classic paper, Hutchinson asked why,
given the apparently homogeneous environment
and small number of limiting resources,
freshwater phytoplankton communities are so incredibly diverse.

Given the parallels between Hutchinson's question and our own,
it does not seem so far fetched that community ecology literature would have
answers relevant to human microbiology.
Along with Hutchinson's earlier treatise,
\article{Homage to Santa Rosalia}~\citeyearpar{Hutchinson1959},
which has a similar theme,
\article{The Paradox} has inspired much of the
last half century of community ecology.
Numerous mechanisms allowing for stable coexistence have been
hypothesized and tested, expanding our understanding of niche
processes.
What's more, challenges to a number of assumptions implicit in our
surprised reaction to this high species richness
have yielded alternative models of community assembly
which may improve our holistic understanding.
Here I will review this literature in order to identify processes which may
contribute to the high diversity of the human gut microbiome.

\section{Explanations for Diversity}
Through a combination of theoretical and empirical study, a number of
processes which oppose competitive exclusion have been identified,
and may contribute to the maintenance of species richness in nature.
These, and perhaps some not yet described,
must ultimately explain the extreme diversity of the human gut.
By understanding the underlying mechanisms,
more detailed explanations for the relationships between microbiome
diversity and human health may be elucidated.

Mechanisms which serve to maintain species richness can be conceptually
partitioned into two categories~\citep{Vellend2010}.
`Niche processes' increase the strength of intraspecific competition
relative to interspecific.
Resource specialization, predator mediated coexistence, and temporal niches
represent just a few;
these and others will be described below.
`Neutral processes', encompass everything that is not dependent on the
fundamental differences between species.
Ecological drift, speciation, and migration are the major neutral processes
and their role in gut diversity will also be explored.

\subsection{Niche Processes}
Competitive exclusion is dependent on the mutual, negative interaction
between two species competing for the same limiting resources.
While these resources are frequently trophic, they need not be:
physical space~\citep{Hastings1980,Paine1971},
inhibitory or toxic byproducts---especially in microbes~\citep{MacLean2008}---%
% TODO: AWKWARD
and other density dependent constraints on growth
can also be limiting resources.
Intriguingly, one or more aspects of host immune response may
play the role of a resource in the microbial community.
For instance, cross-reactivity of the immune system to two members
of the microbiome may lead to mutually negative effects of each
species on the other, analogous to a limiting trophic resource
\citep[For a review of the interaction of the microbiota with the immune
system, see][]{Brestoff2013}.

Simple, non-mechanistic models of multi-species growth,
such as Competitive Lotka-Volterra~\citep{Lotka1925,Volterra1928},
incorporate these limitations as a single term for each species,
the carrying capacity of the environment,
while a separate interaction terms define the effect of the other species
on that carrying capacity.
The carrying capacity terms are therefore a description of the strength
of self-regulation for each species---i.e\@. intraspecific competition---and
the ratios of the interaction terms to carrying capacities can
be interpreted as the strength of interspecific competition.

\subsubsection{Ecological Niches}
When species consume the same limiting resource
interspecific interactions are relatively strong,
whereas without overlap in consumed resources,
interactions may be non-existent
% TODO: Consider discussing Limiting similarity? Character displacement?
Along with our intuition, analysis of this model shows that
two species can stably coexist when
intraspecific effects are stronger than interspecific for both%
~\citep{Chesson2000a}.
Resource niches,
which decrease the magnitude of interaction terms,
therefore promote coexistence.

As an intriguing side-effect of this model,
when interspecific effects are stronger than intraspecific for both
species,
a potential result of allelopathic interactions and
interference competition,
the outcome is dependent on the starting state of the system%
~\citep{Amarasekare2002};
whichever species is initially dominant outcompetes the other.
In the gut microbiome, similar `priority effects' could explain the
temporally stable differences between individuals%
~\citep{Dethlefsen2006,Lahti2014},
or the high recurrence rate of antibiotic associated diarrhea%
~\citep{Lemon2012}.
Additional study of alternative stable states in the human
microbiome could lead to improved treatments for various ailments.

Based on this model of niche mediated coexistence,
it is reasonable to ask if there are enough niches
to explain the thousand or more OTUs co-occurring in the human gut.
Resource niches in heterotrophic macrobes are largely
determined by the size and resistance of prey items;
It's easy to see how a species of finch might specialize on
seeds of particular size and shell thickness.
Heterotrophic bacteria, by contrast, are only capable of
absorbing small molecules,
no larger than several hundred daltons.
Is it possible that there are sufficiently many \emph{different}
substrates with high enough abundance and stability
to sustain each of these microbial populations?
Just like Hutchinson's plankton,
at first glance, it would appear unlikely that the diversity
of limiting resources could match the observed species
diversity.
Nonetheless, this is an empirical question worth pursuing,
and tallying the array of trophic options available in the gut
should yield interesting results.
% TODO: Discuss previous work like this.
One first-pass approach would be to identify the number of
ATP generating metabolic modules in a gut shotgun metagenome%
~\citep{Lemon2012}.
% Reference shotgun metagenomes previously collected.
Even if there are a multitude of carbon sources,
nitrogen compounds may be more broadly limiting~\citep{Faith2011}.
If this is true, the case for a niche based explanation for
gut species richness would appear tenuous
as the diversity of nitrogen sources is relatively low.


While Competitive Lotka-Volterra is able to model the
persistence of species which occupy sufficiently different
niches,
perhaps more satisfyingly,
Tilman's explicitly mechanistic competition model~\citep{Tilman1977},
also yields stable coexistence between species.
By describing not only the growth of multiple populations
but also the depletion of multiple resources,
this model was able to demonstrate that
species dependent on the exact same substrates may coexist
when each is growth limited by a different one.
For instance, two bacteria which both grow on glucose and
ammonium may persist indefinitely if glucose concentrations are
low enough to check the growth of one, and ammonium
concentrations are low enough for the other.

As a result of this mechanism, supplementation of a
limiting resource may remove the constraint on one species'
growth, allowing it to exclude the other in competition for
the remaining limiting resource.
This effect was harnessed by Tilman to show that increasing
resource availability can result in reduced diversity of plant communities%
~\citep{Harpole2007}.
A similar test in mice could asses the importance of limiting
resources in maintaining gut biodiversity.

\subsubsection{Environmental Heterogeneity}
Of course, the entire length of the human digestive system can
hardly be considered a single environment.
Fecal samples, the source of most richness estimates, are the
summation of microbial populations which persist in
food, oral, gastric, small intestinal, and colonal habitats,
and maybe other ecologically relevant subdivisions.
Perhaps the total number of resource niches contributing
to the diversity of microbes in feces,
is better estimated as the sum of limiting resources
within each environment along the digestive tract.
Environmental factors, such as temperature, pH,
oxygen availability, and the host immune system,
all affect%
---and in many cases invert---%
the outcome of competition~\citep{Human2012}.

Oxygen partial pressure is a particularly interesting
component of the host environment, with the potential
to substantially influence both the niches available,
and the outcome of competition in the gut~\citep{Espey2013}.
At the oral end of the gastrointestinal tract, bacterial
communities have access to high concentrations of oxygen,
resulting in selection for those species which are able to
reap the benefits of aerobic respiration
and resist the toxic effects of the oxygen molecule itself~\citep{Zhang2014}.
Respiration by the microbial community quickly depletes
oxygen concentrations until,
after a short distance into the small intestine,
the gut lumen is largely anaerobic.
Host absorption removes simple molecules in the small
intestine,
leaving bacteria in the colon with a
rich array of polysaccharides to be broken into glucose or
other monomeric sugars and fermented~\citep[reviewed in][]{Macfarlane2012}.
Ultimately, this process provides the host with large
quantities of short-chain fatty acidsn,
contributing as much as 10\% of total caloric intake in humans%
~\citep{McNeil1984}.

Despite the anaerobic nature of the intestinal lumen,
the gut epithelium is richly perfused with blood, resulting
in diffusion of oxygen through the mucous layer into
the internal environment.
This proximodistal oxygen gradient suggests that some
gut bacteria, especially those occupying the mucosa,
may be able to respire the abundant sugars and short-chain fatty acids,
a significantly more profitable metabolism.
Clearly, oxygen dynamics alone create a number of
divergent `climates' in the gut,
potentially supplying new niches,
and maintaining the long-term viability of additional
species.

This environmental differentiation between body sites may be
paired with microenvironmental variation at the centimeter scale or smaller%
~\citep{Zhang2014}.
Temporal fluctuations, as well, are probably a ubiquitous
experience for microbes in the human gut due to behavioral and physiological
cycles of the host and directional flow of lumen contents.
One study suggests that the gut microbiome may have its own circadian rhythm%
~\citep{Thaiss2014}.
Both temporal and spatial heterogeneity have been studied as
mechanisms of coexistence~\citep{Chesson2000}.
When environmental changes invert the direction of selection,
the frequency of these fluctuations, whether in space or time,
is expected to have an important effect on the outcome of competition%
~\citep{Hutchinson1961}.

For many species of microbes%
---as in most plants and some animals---%
dormant or resistant life-stages can persist in the environment for long
periods of time~\citep{Lennon2011}.
These `seed banks' have important implications for local diversity,
since they integrate species richness over past environmental conditions%
~\citep{Jones2010}.
Microbial seed banks have been shown to contribute to the diversity found in
soil~\citep{Lennon2011}.
While dormant stages are probably quickly washed out of the linear
reaches of the human gut,
the appendix and other branched portions may provide for the long-term
persistence of slower growing comunity members~\citep{Bollinger2007}.

\subsubsection{Evolutionary Trade-offs}
The observation that the outcome of competition is context
dependent reflects another central theme of community ecology.
No species is ecologically superior in all environments,
demonstrating the universality of physiological trade-offs.
Negative pleiotropy between the ability to grow on
different substrates directly explains resource specialization%
~\citep{Futuyma1988}.
% TODO: Is this statement too strong?
Besides those based on environmental and trophic trade-offs%
~\citep{Litchman2007},
an entire suite of coexistence mechanisms
depend on more abstract correlations between species traits.

One such mechanism is the competitor-colonizer trade-off
(\citealp{Levins1971}; for a brief review see \citealp{Yu2001}),
which results in coexistence through the partitioning of space between species.
Given a trade-off between dispersal or colonization ability
and competitive ability,
a superior colonizer can quickly occupy habitat space which has become open
due to local catastrophe or chance extinction.
While the superior competitor will eventually reach the same locality
and outcompete the first species,
with sufficient density of unoccupied space
the colonizer may be able to disperse quickly enough to maintain a stable
population.
Depending on the particular parameters,
these `fugitive' species can coexist indefinitely with the superior competitor.
It is possible that a similar model describes successional processes
on food particles in the gut~\citep{Macfarlane2006}.

\subsubsection{Network Effects}
While negative frequency dependent effects,
such as resource limitation,
serve to stabilize two-species interactions,
third species can also contribute to these negative feedbacks,
sometimes promoting stable coexistence.
Predator and parasite mediated coexistence have been described
theoretically~\citep{Caswell1978}
and empirically~\citep{Paine1966}.
% TODO: and many others, see Caswell
The result of a new exploitative interaction on the initial
competition is partially dependent on the exact structure of the
trophic network;
either a generalist predator or one which specializes on the weaker
competitor serves to quicken the rate of ecological selection.
Conversely, independent predators of each species or a predator
which specializes on the stronger competitor can lead to stable
coexistence.
In part due to the popularity of 16S based surveys of the
microbiome
which often ignore both eukaryotic
and viral members of the community,
studies of predators and parasites of bacteria and archaea
in the human gut are lacking.
Incorporating these factors may help to explain some of the
remarkable biodiversity.

Undoubtedly portions of the species richness observed in macroscopic
organisms can be explained not through exploitative,
but rather mutualistic interactions.
Pollinator species in the tropics, for instance, frequently
have highly specific mutualisms with flowering plants which
provide trophic resources~\citep{Bawa1990}.
The specificity of this relationship provides tailored niches
for individual pollinators, potentially contributing substantially
to the total species capacity of these ecosystems%
~\citep{Burger1981,Stebbins1981}.
Analogous mutualisms in microbes, for instance
between bacteria which produce auto-inhibitory hydrogen
and methanogenic archaea~\citep{Narihiro2014},
% TODO: Is this an appropriate citation?  Talk to Jessica.
could similarly support increased diversity.
High specificity mutualisms are likely responsible for some of the species
richness found in the gut,
but exactly how much is unclear.
Further research into these types of species interactions
may inform not only our understanding of biodiversity,
but also provide a predictive framework for the impact of
perturbations on community composition.

\subsection{Neutral Processes}
The search for mechanisms of coexistence has been a near obsession for
community ecologists.
This is partially a byproduct of the popular analytic approaches,
especially stability analysis,
many of which were brought to the field from mathematics
in the middle of the 20th century~\citep[e.g\@.][]{May1973,MacArthur1955}.
More recent influences have come from thermodynamics,
statistical mechanics, and evolutionary biology,
inspiring a new wave of non-equilibrium theories of biodiversity.
Here we will refer to these collectively as `non-niche' or `neutral'
processes,
although this means grouping a diverse array of forces by only their
distinction from niche mechanisms.
Three general processes fall into this category:
drift, fluctuations in the abundance of finite populations due to
the stochasticity of survival and reproduction;
migration, the movement of individuals into or out of the system;
and speciation, the in situ generation of diversity through
evolutionary change.
Along with the niche process of ecological selection,
these perfectly parallel the four processes considered in population genetics%
~\citep{Vellend2010}:
selection, drift, mutation, and migration.
Neutral processes have major implications for the maintenance of
biodiversity in the human gut.

\subsubsection{Ecological Drift}
Ecological drift, like its evolutionary counterpart, is the result
of the inherently statistical nature of births and deaths in
natural populations, and results in a random walk of
relative abundances through time.
Like competitive exclusion, acting in isolation for
sufficient time,
ecological drift results in the loss of diversity.
This is because the random walk ultimately
leads to extinction, a non-reversible event.
Whereas niche processes are deterministic and standard mathematical
models describe dynamics in infinitely large communities,
ecological drift is strongest in small populations, where
stochastic births, deaths, and disasters can result in large
percent changes in abundance.
Microbes of detectable abundance in the human gut are necessarily
large populations.
With on the order of \(10^{14}\) microbial cells in the human GI tract%
~\citep{Whitman1998},
any species at even 0.01\% abundance in community surveys
has a population size akin to the number of people on the planet.
Nonetheless, even large populations are subject to drift,
especially in this case,
since the physical locations of a particular species in the gut lumen is likely
clumpy~\citep{Green2007},
and therefore death and washout may not be statistically independent events.
Any finite size community is therefore not in equilibrium,
and consideration of long-term stability is superseded by
analysis of statistical expectations over time.
This perspective is a reminder that community composition,
and therefore species richness, is constantly in flux.

The stochasticity of ecological drift is tempered, however,
by the determinism of niche processes,
which in some cases buffer species from extinction through
negative density or frequency dependent selection,
and in others quickens the pace through unmitigated competition.
% TODO: Unmitigated?
Besides small population sizes, the relative importance of
drift is increased when species are of more similar fitness.
Mechanisms which decrease fitness differences can therefore
increase the time to competitive exclusion~\citep{Chesson2000,Adler2007}.

One such equalizing mechanism is environmental variability.
When climatic variables fluctuate at a frequency similar to the
rate of competitive exclusion, such that the relative fitness of
two competitors is continually inverted,
fitness difference are effectively brought closer to zero%
~\citep{Hutchinson1961,Chesson1981}
% TODO: Is the Chesson ref correct?
Short term diversity can therefore be increased by
environmental fluctuations at the right time scale.
In the human gut, this mechanism may be highly important,
since temporal fluctuations are ubiquitous and occur at a variety of time
scales~\citep{Thaiss2014}.
Alternatively, even in static environments,
fitness equivalency may be the norm,
a hypothesis supported by the prevalence of physiological trade-offs%
~\citep{Adler2007}.

Whether the diversity of the human gut is the result of
deterministic, stabilizing mechanisms, or processes which equalize
fitness between species is a key question~\citep{Adler2007},
and may have implications for the management of microbiome related
dysbiosis.

\subsubsection{Migration and Speciation}
With competition and drift actively removing diversity from local
communities, migration and speciation serve to as a counterbalance,
introducing new species either \latin{de novo},
in the case of speciation, or by immigration from outside the
local system.
By conceptually partitioning local and regional communities,
speciation within the relatively small, local community can be effectively
ignored and replaced with migration,
though speciation at the regional scale is the ultimate source of global
biodiversity.
The Theory of Island Biogeography~\citep{macarthur1967b} was
% Is this the correct reference?
an early model describing the effect of migration rates on local diversity,
specifically the balance of two rate processes:
migration from a external species pool and local extinction.
Surprisingly, this simple, equilibrium model of species richness
fits well patterns of diversity on island archipelagos
% TODO: "fits well"?
without invoking niches or competitive differences~\citep{Simberloff1976}.
Island Biogeography, as a model of community assembly in the human gut,
predicts a constantly changing species composition,
despite an equilibrium species diversity.
This is not held up by the empirical observation that the gross composition of
the human gut microbiome is relatively constant.
Regardless, a portion of the members of the gut community may be determined
by migration from the external environment.
Searching 16S time series for these kinds of dynamics may be an important
line of research.

Stephen Hubbel's Unified Neutral Theory of Biodiversity%
~\citeyearpar{Hubbell2001}
attempted to replicate this success,
but with a more explicit treatment of mechanism
and several controversial assumptions which limit applicability.
Hubbel's model has been
reformulated for microbial populations~\citep{Sloan2005},
and may serve as a valuable null hypothesis in identifying community
compositions which cannot be explained
by purely neutral processes~\citep{VenkataramenTODO}.
It is possible that
a significant fraction of the species in the human gut are
transients from the external environment.
If this hypothesis is correct then community composition should reflect
primarily neutral processes.

\subsection{Emergent Properties and Higher Level Selection}
As a result of the immense importance of the gut microbiome to host health,
the global prevalence of any single species is not only determined by its
relative fitness in competition with other community members,
but also its effect on fitness of the host.
This is a classic example of an extended phenotype~\citep{Dawkins1982},
and may help to explain
the evolution of host-microbe mutualisms~\citep{Dethlefsen2007a}.
Some of the benefits conferred by microbial communities
are likely to be not just the sum of its parts,
but irreducible properties of the community structure as a whole.
Community stability and invasion resistance~\citep{Levine1999},
for instance, may protect the host from pathogens~\citep{Lozupone2012},
and is not yet been shown to be a property of any individual symbiont%
~\citep{Cordero2014}.
Diversity itself may be causally related to both community function and
stability~\citep[reviewed in][]{McCann2000}.
Selection for emergent properties which are most advantageous to the host,
combined with high fidelity transmission of communities between hosts,
sets the stage for evolution to act not at the level of microbial populations,
but instead on full communities~\citep{Swenson2000,Ley2006}.
Higher level selection could result in the evolution of a highly diverse
communities~\citep{MacArthur1955}.
Alternatively, host genotypes which promote community diversity may be
selected for,
a reversed example of an extended phenotype.

Hypotheses which depend on higher level selection are open to several common
criticisms~\citep{Wade1978}.
Foremost among these is the necessary fidelity of transmission.
If communities are assembled from a largely random assortment of environmental
microbes,
then selection acting on emergent properties does not affect the prevalence
of these community level traits over time.
Alternative explanations, such as monotonic effects of symbionts on host
fitness and host genotype control are often more parsimonious.
% TODO: Does this need a citation?

\section{Conclusion}
With little previous research, the source of extreme species richness
in the human gut is a standing question.
The relative importance of stabilizing mechanisms which promote
long term coexistence, versus non-equilibrium processes, such as
drift, migration, and \latin{in situ} diversification remains to be tested,
and all four processes are likely to take part.
For community ecology, Hutchinson's \article{Paradox of the Plankton} has
served as a focus for theoretical study,
ultimately leading to a diverse mathematical basis for empirical
study.
With the advent of culture-free methods in microbiology,
human microbiome research has quickly assumed the task of describing
our symbiotic community.
As the field begins the process of developing predictive capabilities
and tools for community management,
a similarly robust theoretical foundation will be of paramount
importance.
A central question, such as explaining the observed
diversity of the gut community, will serve to motivate and focus
theory and experimentation, with positive repercussions for
applications as well.
Human microbiome research \emph{is} community ecology.
While there are undoubtedly qualitative differences between
microbial and macroscopic communities, the literature of traditional
ecology is a valuable resource in the transition to a predictive
and applied science.

\bibliographystyle{myplainnat}
\bibliography{master.bib}

\end{document}
